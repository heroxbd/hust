\documentclass[a4paper,conference]{IEEEtran}

\usepackage[utf8]{inputenc}
\usepackage{hyperref}

\hypersetup{
  pdfauthor = {Guilherme Amadio and Benda Xu},
  colorlinks, urlcolor=black, citecolor=black,
  pdftitle = {Portage: Bringing Hackers' Wisdom to Science},
  pdfsubject = {Third International Workshop on HPC User Support Tools}
}

\title{Portage: Bringing Hackers' Wisdom to Science}

\author{
  \IEEEauthorblockN{Guilherme Amadio}
  \IEEEauthorblockA{São Paulo State University, Brazil\\
                    Gentoo Linux\\amadio@gentoo.org}
  \and
  \IEEEauthorblockN{Benda Xu}
  \IEEEauthorblockA{The University of Tokyo, Japan\\
                    Gentoo Linux\\heroxbd@gentoo.org}
}

\begin{document}
\maketitle

\begin{abstract}

Providing users of HPC systems a wide variety of up to date software
packages is a challenging task. Large software stacks built from source
have difficult to manage dependencies, requiring powerful package
management tools. The Portage package manager from Gentoo is a highly
flexible tool that offers a mature solution to this otherwise daunting
task. The Gentoo Prefix project develops and maintains a way of
installing Gentoo systems in non-standard locations, bringing the
virtues of Gentoo to other operating systems. Here we shall demonstrate
how Gentoo Prefix installations can be used to cross compile software
packages for the Intel® Xeon Phi™, as well as to manage large software
stacks in HPC environments.

\end{abstract}

\section{Introduction}

Building and maintaining a large software stack on an HPC environment
requires powerful packaging tools. Packages provided by the core operating
systems from HPC vendors often lag significantly behind the latest
upstream releases, while users want to make use of new features in
recent software releases, or build against specific versions of software
packages for compatibility or consistency reasons. In order to satisfy
users, the package manager must take into account the multitude of
configuration options for each package as well as ensure that the build
order is correct. It must also be able to distinguish between build
time, run time, and post build dependencies, and be able to force
package rebuilds when configuration options of dependencies change,
without rebuilding the full tree of packages. Finally, due to the
particularities of each HPC environment, it is highly desirable for the
package manager to offer the ability to easily apply patches to specific
software packages. Fortunately, Gentoo's \cite{gentoo} Portage
\cite{gentoo:portage} package manager is able to do all this and much
more. On HPC systems where users usually do not have permission to
install software into standard locations, Gentoo Prefix
\cite{gentoo:prefix} allows installations of a Gentoo environment into
non-standard paths.

\subsection{Gentoo Linux}

Gentoo Linux \cite{gentoo} is a source-based distribution with powerful
tools that make it an excellent development environment. Gentoo has been
first released by its founder Daniel Robbins in December of 1999, and
has been in active development since then. It is a metadistribution, in
the sense that it provides the necessary tools for the user to build
his[her] own customized version of Gentoo. Gentoo uses its own package
manager, Portage \cite{gentoo:portage}, written in Python and inspired
by the ports system from FreeBSD. Gentoo is not based on any Linux
distribution, but is itself the root of popular distributions such as
Arch Linux, Sabayon Linux, and innovative operating systems such as
CoreOS \cite{coreos} and Google's ChromeOS \cite{chromiumos}.

\subsection{Gentoo Prefix Project}

The Gentoo Prefix project \cite{gentoo:prefix} brings the virtues of
Gentoo Linux---such as high configurability, performance tuning, and
automated package dependency management---to different operating
systems. This is useful for installing software into systems where the
user might not have administrator privileges, or simply to use Gentoo's
Portage package manager to automate the process of fetching source code,
building, and installing software packages available in Gentoo and
having their dependencies automatically handled, including the
configuration options of each package.

Gentoo Prefix uses the host system's kernel and standard C library, but
all other tools are installed and managed during the bootstrapping of
the system. There is also a more experimental version of Gentoo Prefix
that compiles its own standard C library. Installation of the Gentoo
Prefix environment is performed by a shell script that installs the
Portage package manager into a temporary location, then uses it to
bootstrap a compiler and install the base system, which allows the user
to compile and install other sofware packages available in the Portage
database. Gentoo Prefix can be installed not only on Linux, but also on
Mac OS X, BSD, AIX, Solaris, etc. Gentoo Prefix is similar to what
projects such as Homebrew and MacPorts provide for Macs, but builds
its packages from source, rather than offering binary packages.

\subsection{Portage Package Manager}

Portage is a GPLv2 package management system based on FreeBSD's ports
collection. Portage consists of two main parts, the \emph{ebuild}
system, and the \emph{emerge} command line utility. The ebuild system is
responsible for executing build instructions and installing packages,
while emerge provides dependency management, and servers as the
interface to ebuild. The relationship between ebuild and emerge is not
unlike that of rpm and yum on Red Hat, and that of dpkg and apt on
Debian.

Portage packages are special bash shell scripts called \emph{ebuilds}
(not to be confused with the ebuild tool itself that is used to run
them). Ebuilds are similar to spec files in SRPMs; they contain
information on how to download, configure, compile, and install
software, as well as their dependency requirements. Gentoo's main ebuild
repository has more than 27000 ebuilds available for a variety of
architectures. Additional packages are available via official and
unofficial package overlays that complement the main tree of packages.
The special syntax used in ebuild scripts is standardized in the Package
Manager Specification (PMS) \cite{gentoo:pms} document. The PMS
documents the behavior of Portage so that Gentoo packages can be managed
by alternative package managers.

The Portage system has the concept of \emph{USE flags} to allow the user
to configure compile-time options of software packages. USE flags affect
which dependencies are required to build a package, which allows, for
example, a headless server to be installed with a lighter system image,
by stripping all options for building a graphical environment. USE flags
can be used in HPC systems to enable or disable MPI support, to choose
which version of the Python interpreter a module should target, to
enable specific instruction sets for a given architecture, among many
others.

For software packages that require specific versions, Portage has the
concept of package \emph{SLOTs}, which allow multiple versions of a same
package to be installed simultaneously on the same system. This is very
useful for, e.g., installing packages that depend on incompatible
versions of the same library, or to have multiple compiler versions
available for development and testing. SLOTs can also be used in package
dependencies to indicate when a package should be rebuilt should the
installed version of its dependencies change.

% TODO: keywords, stable/unstable/masked, patching, profiles.

\section{Cross Compiling Software for the Xeon Phi™}

\section{Gentoo Prefix on an HPC environment}

\bibliographystyle{IEEEtran}
\bibliography{portage}
\end{document}
